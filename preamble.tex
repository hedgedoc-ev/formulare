%%%%%%%%%%%%%%%%%%%%%%%%%%%%%%%%%%%%%%%%%%%%%%%%%%%%%%%%%
%%%               Beginn der Präambel                 %%%
%%%%%%%%%%%%%%%%%%%%%%%%%%%%%%%%%%%%%%%%%%%%%%%%%%%%%%%%%

%%%%%%%%%%%%%%%%%%%%%%%%%%%%%%%%%%%%%%%%%%%%%%%%%%%%%%%%%
%%%                     Pakete                        %%%
%%%%%%%%%%%%%%%%%%%%%%%%%%%%%%%%%%%%%%%%%%%%%%%%%%%%%%%%%
\usepackage{scrlayer-scrpage}
\usepackage[pdftex,colorlinks=true,linkcolor=black]{hyperref}
\usepackage[T1]{fontenc}
\usepackage[default]{sourcesanspro}
\usepackage{multicol}
\usepackage{microtype}
\usepackage{graphicx}
\usepackage{xcolor}
\usepackage{minibox}
\usepackage{multicol}
\usepackage{tikz}
\usepackage{lastpage}

\ofoot*{\normalsize \emph{\thepage\nobreakspace von\nobreakspace \pageref{LastPage}}} 
\cfoot*{HedgeDoc e.V. $\bullet$ Pulverstraße 1 $\bullet$ 44225 Dortmund}

\usepackage[pdftex,colorlinks=true,linkcolor=black]{hyperref}

%%%%%%%%%%%%%%%%%%%%%%%%%%%%%%%%%%%%%%%%%%%%%%%%%%%%%%%%%
%%%           Änderungsdaten der Ordnungen            %%%
%%%%%%%%%%%%%%%%%%%%%%%%%%%%%%%%%%%%%%%%%%%%%%%%%%%%%%%%%

\newcommand{\datumsatzung}{17.03.2022 }
\newcommand{\datumbeitrago}{23.02.2022 }

%%%%%%%%%%%%%%%%%%%%%%%%%%%%%%%%%%%%%%%%%%%%%%%%%%%%%%%%%
%%%    Befehle und Definitionen der Formularfelder    %%%
%%%%%%%%%%%%%%%%%%%%%%%%%%%%%%%%%%%%%%%%%%%%%%%%%%%%%%%%%

%% Veränderung der Standard-Einstellungen der Formularfelder
\renewcommand*{\DefaultOptionsofText}{%
print,
borderstyle=U, % S = Solid (default), B = Beveled, D = Dashed, I = Inset und U = Underlined
bordercolor=0 0 0, % 0 0 0 = schwarz und 1 1 1 = weiß
borderwidth=0.4pt,
}

\renewcommand*{\DefaultOptionsofCheckBox}{%
print,
checkboxsymbol=4,
height=0.8\baselineskip, % dieser Wert ist eigentlich default, aber irgendwie geht das kaputt wenn man das nicht hier setzt
borderstyle=S, % S = Solid, B = Beveled, D = Dashed, I = Inset und U = Underlined
bordercolor=0 0 0, % 0 0 0 = schwarz und 1 1 1 = weiß
borderwidth=0.4pt,
}

%% Veränderung der Platzierung der Beschriftung der Formularfelder
\renewcommand{\LayoutTextField}[2]{\minibox[t]{#2 \\{\footnotesize #1}}} 
\renewcommand{\LayoutCheckField}[2]{\minibox[t]{#2 \ #1}} 
\renewcommand{\LayoutChoiceField}[2]{\minibox[t]{\hspace*{-1em} #2 \ #1}}

%% Zähler für das Label name= der Textfelder 'tf' und CheckBoxen 'cb'
\newcounter{tf}
\newcounter{cb}

%% Definition der eigenen Befehle zur einfachen Erstellung der Formularfelder
\newcommand{\tfw}[2]{\TextField[name=\thetf,width=#1]{#2} \stepcounter{tf}} % 'text field with width' TextField mit Angabe der Breite 'width'
\newcommand{\cb}[1]{\CheckBox[name=\thecb]{#1} \stepcounter{cb}} % CheckBox
\newcommand{\cm}[2]{\ChoiceMenu[%
radio,
radiosymbol=\ding{108},
print,
height=0.8\baselineskip,
borderstyle=S,
bordercolor=0 0 0, % 0 0 0 = schwarz und 1 1 1 = weiß
borderwidth=0.4pt,
name=#1]{#2}{\quad}} % ChoiceMenu mit dem Symbol 'radio' und voreingestellter Höhe

%% Achtung! Kein Formularfeld: Linie für handschriftliche Eintragungen wie z.B. Unterschriften
\newcommand{\hwf}[2]{\minibox[t]{{\rule{#1}{0.4pt}}\\[-5pt]{\footnotesize #2}}} % 'hand-written field' - Handschriftliches Feld für z.B. Unterschriften mit Angabe der Breite
